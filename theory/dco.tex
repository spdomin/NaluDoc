An alternative to classic Peclet number blending is the usage of a discontinuity capturing operator (DCO).
In this method, an artifical viscosity is defined that is a function of the local residual and scaled computational 
gradients. Viable usages for the DCO can be advection/diffusion problems in addition to the aforementioned RTE
VMS approach.

The formal finite element kernel for a DCO approach is as follows,
\begin{equation}
  \sum_e \int_\Omega \nu(\mathbf{R}) \frac{\partial w}{\partial x_i} g^{ij} \frac{\partial \phi} {\partial x_j} d\Omega,
\label{dcoFEMForm}
\end{equation}
where $\nu(\mathbf{R})$ is the artifical viscosity which is a function of the pde fine-scale residual and $g^{ij}$ 
is the covariant metric tensor.

For completeness, the covariant and contravarient metric tensor are given by,
\begin{equation}
  g^{ij} = \frac{\partial x_i} {\partial \xi_k}\frac{\partial x_j} {\partial \xi_k},
\label{coVariant}
\end{equation}

 and

\begin{equation}
  g_{ij} = \frac{\partial \xi_k} {\partial x_i} \frac{\partial \xi_k} {\partial x_j},
\label{coVariant}
\end{equation}

where $\xi = (\xi_1, \xi_2, \xi_3)^T$.
%
The form of $\nu(\mathbf{R})$ currently used is as follows,
\begin{equation}
  \nu = \sqrt{ \frac{\mathbf{R_k} \mathbf{R_k}} {\frac {\partial \phi}{\partial x_i} g^{ij} \frac{\partial \phi}{\partial x_j}} }.
\label{nuOne}
\end{equation}

A residual for a classic advection/diffusion/source pde is simply the fine scale residual computed at the gauss point,
\begin{equation}
 \mathbf{\hat R} = \frac{\partial \rho \phi}{\partial t} + \frac{\partial}{\partial x_j} (\rho u_j \phi - \mu^{eff} \frac{\partial \phi}{\partial x_j}) -S
 \label{dcoResidual}
\end{equation}
Note that the above equation requires a second derivative whose source is the diffusion term. The first derivative 
is generally determined by using projected nodal gradients. As will be noted in the pressure stabilization section, 
the advection term carries the pressure stabilization terms. However, in the above equation, these terms are not 
explicity noted. Therefore, we subtract the fine scale continuity equation to obtain the final general form of 
the source term,
\begin{equation}
 \mathbf{R} = \mathbf{\hat R} -  (\frac{\partial \rho}{\partial t} + \frac{\partial \rho u_j }{\partial x_j}).
 \label{dcoResidual}
\end{equation}

In general, the DCO-$\nu$ is prone to percision issues when the gradients are very close to zero. As such, the 
value of $\nu$ is limited to a first-order like value. This parameter is proposed as follows: if an operator
were written as a Galerkin (un-stabilized) plus a diffusion operator, what is the value of the diffusion coefficient
such that first-order upwind is obtained for the combined operator? This upwind limited value of $\nu$ provides the
highest value that this coefficient can (or should) be. The current form of the limited upwind $\nu$ is as follows,

\begin{equation}
  \nu^{upw} = C_{upw}(\rho^2  u_i g_{ij} u_j )^{\frac{1}{2}}
\label{dcoFVForm}
\end{equation}
where $C_{upw}$ is generally taked to be ~0.1.

Using a piecewise-constant test function suitable for CVFEM and EBVC schemes (the reader is refered to 
the VMS RTE section), Eq.~\ref{dcoFEMForm} can be written as,
\begin{equation}
  -\sum_e \int_\Gamma \nu(\mathbf{R}) g^{ij} \frac{\partial \phi} {\partial x_j} n_i dS.
\label{dcoFVForm}
\end{equation}
%
A fourth order form, which writes the stabilization as the difference between the Gauss-point gradient 
and the projected nodal gradient interpolated to the Gauss-point, is also supported,
\begin{equation}
  -\sum_e \int_\Gamma \nu(\mathbf{R}) g^{ij} (\frac{\partial \phi} {\partial x_j} - G_j \phi ) n_i dS.
\label{dcoFVForm4th}
\end{equation}

\subsection{Local or Projected DCO Diffusive Flux Coefficient}
While the DCO kernel is certainly evaluated at the subcontrol surfaces, the evaluation of $\nu$ can be 
computed by a multitude of approaches. For example, the artificial diffusive flux coefficient 
can be computed locally (with local residuals and local metric tensors) or can be projected 
to the nodes (via an $L_{oo}$ or $L_2$ projection) and re-interpolated to the gauss points. 
The former results in a sharper local value while the later results in a more filtered-like value.
The code base only supports a local DCO $\nu$ calculation.

\subsection{General Findings}
In general, the DCO approach seems to work best when running the fourth-order option as the second-order 
implementation still looks more diffuse. When compared to the standard MUSCL-limited scheme, the DCO
is the preferred choice. More work is underway to improve stabilization methods.
Note that a limited set of DCOs are activated in the code base with specific
interest on scalar transport, e.g, mixture fraction and static enthalpy transport. When using the $4^{th}$ 
order method, the consistent mass matrix approach for the projected nodal gradients is supported.
